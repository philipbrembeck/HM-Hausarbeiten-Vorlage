\section{Durchführungsteil}

Lorem upsurge dolor sit atmet Lorem up£surge dolor sit atmet Lorem upsurge dolor sit atmet Lorem upsurge dolor sit atmet Lorem upsurge dolor sit atmet Lorem upsurge dolor sit atmet Lorem upsurge dolor sit atmet Lorem upsurge dolor sit atmet Lorem upsurge dolor sit atmet Lorem upsurge dolor sit atmet Lorem upsurge dolor sit atmet Lorem upsurge dolor sit atmet Lorem upsurge dolor sit atmet Lorem upsurge dolor sit atmet Lorem upsurge dolor sit atmet Lorem upsurge dolor sit atmet Lorem upsurge dolor sit atmet Lorem upsurge dolor sit atmet Lorem upsurge dolor sit atmet Lorem upsurge dolor sit atmet Lorem upsurge dolor sit atmet Lorem upsurge dolor sit atmet Lorem upsurge dolor sit atmet Lorem upsurge dolor sit atmet Lorem upsurge dolor sit atmet Lorem upsurge dolor sit atmet Lorem upsurge dolor sit atmet Lorem upsurge dolor sit atmet Lorem upsurge dolor sit atmet Lorem upsurge dolor sit atmet Lorem upsurge dolor sit atmet Lorem upsurge dolor sit atmet Lorem upsurge dolor sit atmet Lorem upsurge dolor sit atmet Lorem upsurge dolor sit atmet Lorem upsurge dolor sit atmet Lorem upsurge dolor sit atmet Lorem upsurge dolor sit atmet Lorem upsurge dolor sit atmet 

\subsection{Durchführungsteil 1.2}
Ein Beispieltext (vgl. \cite{samplebook})
\subsection{Durchführungsteil 1.3}
\enquote{Lorem upsurge dolor sit atmet Lorem upsurge dolor sit atmet Lorem upsurge dolor sit atmet Lorem upsurge dolor sit atmet} \parencite[S. 55]{sample:2}
\subsubsection{Durchführungsteil 1.3.1}
\blockquote{\enquote{Die Harvard-Zitierweise unterscheidet den Kurz- und den Langbeleg. Der knappe Beleg in Ihrem eigenen Text heißt Kurzbeleg. Damit geben Sie an, welches Werk von welchem Autor und aus welchem Jahr Sie benutzen. Er ist sowohl beim wörtlichen Zitat und der sinngemäßen Wiedergabe, also dem indirekten Zitat oder der Paraphrase, nötig. In beiden Fällen geben Sie an, von wem ein von Ihnen verwendeter Gedanke stammt.} \parencite{samplesite:2021}}